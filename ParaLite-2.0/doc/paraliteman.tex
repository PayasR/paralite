\input texinfo  @c -*-texinfo-*-
@setfilename paraliteman.info
@settitle ParaLite User's Guide
@iftex
@setchapternewpage on
@end iftex
@comment %**end of header

Copyright 2012 Ting Chen

@titlepage
@title ParaLite User's Guide
@subtitle May 2012
@author Ting Chen
@author
@author University of Tokyo
@author 7-3-1 Hongo Bunkyo-ku Tokyo, 113-0033 Japan
@page
@vskip 0pt plus 1filll
@end titlepage

@contents

@node Top, Getting Started, (dir), (dir)

@ifhtml
@c HTML <link rel="STYLESHEET" href="gxpman.css" type="text/css" />
@c HTML <h1>GXP3 User Manual</h1>
@end ifhtml

@menu
* Getting Started::             
* Tutorial::                    
* Using ParaLite in Real-World Workflows::  

@detailmenu
 --- The Detailed Node Listing ---

Getting Started

* Prerequisites::
* Installation::                

Tutorial

* Data exchanging methods::
* Performing general SQL query::   
* Performing collective query::  
* Special Command::
* Quick Reference of Frequently Used Commands::  

@end detailmenu
@end menu

@node Getting Started, Tutorial, Top, Top
@chapter Getting Started


@menu
* Prerequisites::               
* Installation::                
@end menu

@node Prerequisites, Installation, Getting Started, Getting Started
@section Prerequisites

To play with ParaLite, you need:
@itemize
@item Python interpreter
@item GXP (Grid and Cluster Shell)
@item SQLite
@end itemize

ParaLite is developed on Linux in python and only tested on Linux. ParaLite supports python 2.7 and probably newer versions, but python 2.5 is 
not supported.

It based on SQLite, a popular single-node database system. 
@example
    http://www.sqlite.org
@end example
You have to install SQLite on each data node.

GXP is used to explore nodes. It is a parallel shell tool to let you 
run an identical or a similar command line to many machines in parallel 
and get results back interactively. You can obtain it from sourceforge:
@example
    http://sourceforge.net/projects/gxp/
@end example

@node Installation,  , Prerequisites, Getting Started
@section Installation

unpack the tarball:
@example
    $ tar zxvf paralite-xx.tar.bz2
@end example
add @t{paralite} to the execution path:
@example
    $ ln -s /path/to/paralite-xx/paralite /usr/local/bin/paralite
@end example

or,
@example
    $ export PATH=$PATH:/absolute/path/to/paralite-xx
@end example

To test your installation, type @t{paralite} to your shell prompt and see
something like this.
@example
$ paralite
  Usage: paralite /path/to/db STATEMENT [--hub]
  Please enter "paralite help" for more information.
@end example
Before a query is issued, you should assure that all related nodes have already
been explored by GXP.
@node Tutorial, 
@chapter Tutorial

@menu
* Data exchanging methods::
* Performing general SQL query::   
* Performing collective query::  
* Special Command::
* Quick Reference of Frequently Used Commands::  
@end menu

@section Processes coordination methods

ParaLite is server-less and zero-configuration system, that is, you don't need
start any process and specify any configuration before SQL is executed. To achieve this,
a super process should be started to coordinate all other processes after they are started. 
Then where to start the super process and how everybody knows it? You need to specify
hub information at the end of each query using three methods:

(1) @t{--hub db:hub_db} : Using a database as a hub;

@example
$ paralite /path/to/db "query" --hub db:hub_db
@end example
If no hub info is specified, paraLite uses @t{/path/to/db} right after @t{paralite}
as default hub. 

(2) @t{--hub file://path/to/file} : Using a separate file 
which can be opened by all processes;

(3) @t{--hub host://hostname:port} : Starting the master process that listens 
to the @t{port} on machine @t{hostname}. This is commonly used because you
can specify the super process running on a machine that explored all related nodes.

Note that, (1) or (2) is used only if the database and file can be shared among all 
processes, e.g. NFS is used. 

@section Performing general SQL query

First, you can create table by the following command:
@example
$ paralite test.db "create table x(a int, b varchar(10))"
@end example
This query is to create table @t{x} locally. If you want to specify the 
data nodes on which the table is created,
@t{on} option is required as the following query:
@example
$ paralite test.db "create table x(a, b) on huscs000"
@end example
Several nodes can be specified in a single query delimited by space.

If you have many nodes, typing each node name is very troublesome and painful.
An improved description for multiple nodes is 
@t{[[xxx-yyy]]}, which represents a set of numbers between @t{xxx} and @t{yyy} 
(inclusive). 
@example
paralite test.db "create table x(a, b) on huscs[[000-002]]"
@end example
You can also use a configuration file to specify data nodes:
@example
$ paraLite test.db create table x(a, b) on file node.conf
$ cat node.conf
huscs[[000-003]]
hongo[[100-112]]
@end example

This table can be partitioned either by hash fashion based on a specific 
key or round-robin fashion (by default). 
You can also specify the number of replica for 
each partition, 1 by default.
@example
$ paralite test.db "create table x(a, b) [partition by key] [replica 3]"
@end example

Then, you need to load data to database:
@example
$ cat test.dat
aa|test1
bb|test2
aa|test3
$ paralite test.db ".import test.dat x"
@end example
The default row separator and col separator in ParaLite are @t{\n} and @t{|}
respectively. If your source data is not separated by them, assuming that row 
separator is '===' and col separator is '###', you can either 
change them by 
@example
$ paralite test.db ".row_separator ==="
$ paralite test.db ".col_separator ###"
@end example
or specify them in the @t{.import} command 
@example
$ paralite test.db ".import test.dat x -column_separator ### -row_separator ==="
@end example
Note that, at this point, paraLite cannot support session, that is, the values of all
settings are changed explicitly.

Next, you can perform some selection on the table:
@example
$ paralite test.db "select * from x where a='aa'"
aa|test1
aa|test3
$ paralite test.db "select count(*) from x"
3
$ paralite test.db "select a, count(*) from x group by a"
aa|2
bb|1
@end example
Currently, ParaLite has some limitation in general SQL as follows:
@itemize
@item (1) At most one selected column is Aggreation Function or UDX. 
E.g. it cannot support SQL like ``select sum(*), count(*) from ..''
@item (2) ParaLite cannot output an column which is an argument of a UDX. E.g. 
select id, F(id) from t with F ...    
@item (3) ParaLite cannot support compound operator: union, intersect, except
@end itemize
These limitations will be fixed in the next version.

@section Performing collective query 
collective query extends SQL syntax in which a user can define User-Defined
Executable (UDX). The syntax of collective query with UDX is:
@example
select a, F(b) as bb, c, ... from t where ....
with F="cmd_line"
     input stdin input_row_delimiter NEW_LINE input_col_delimiter NULL
     output stdout output_row_delimiter NEW_LINE output_col_delimiter NULL 
     output_record_delimiter EMPTY_LINE
collective by 1
@end example

Collective query has two features:

(1) Supportive of User-Defined Executable (UDX): an UDX is an executable binary or
scrip which is defined in the query as a command line. Following the command line,
you may want to specify the format of input and output data.
@itemize
@item @t{input} : stdin by default or '/path/to/file' 
@item @t{input_row_delimiter} : NEW_LINE by default or 'any_string' 
@item @t{input_col_delimiter} : NULL by default or 'any_string' 
@item @t{output} : stdout by default or '/path/to/file' 
@item @t{output_row_delimiter} : NEW_LINE by default or 'any_string' 
@item @t{output_col_delimiter} : NULL by default or 'any_string' 
@item @t{output_record_delimiter} : EMPTY_LINE by default or 'any_string' 
@end itemize
You can understand all these options by the following example:
@example
$ paralite test.db "select * from document"
@t{document_id | text}
32819 | It is sunny today. I have a good mood.
82718 | I am studying in the lab. I want to go outside and play pingpong.
@end example
A executable @t{sentence_splitter} (@t{ss} for short) is used to split a 
text into sentences and add an identification to each sentence. 
@example
$ cat a.dat
Sentence1. Sentence2. Sentence3.
$ cat a.dat | ss 
1==Sentence1.
2==Sentence2.
3==Sentence3.
$ paralite test.db "select document_id, F(text) from document with F=\"ss\" output_row_delimiter EMPTY_LINE"
32819 | 1 == It is sunny today.
        2 == I have a good mood.
82718 | 1 == I am studying in the lab.
        2 == I want to go outside and play pingpong.
$ paralite test.db "select document_id, F(text) from document with F=\"ss\" output_col_delimiter '==' output_record_delimiter EMPTY_LINE"
32819 | 1 | It is sunny today.
32819 | 2 | I have a good mood.
82718 | 1 | I am studying in the lab.
82718 | 2 | I want to go outside and play pingpong.
@end example

An UDX can have more than one argument and output more than one column, e.g. F(a) as 
aa, G(a, b) as bb, G(a, b, c) as(bb, cc).

(2) Identification of Collective Query: 
Each query has an identification specified by @t{collective by ID}. Computing clients
are grouped based on the ID. If a calculation is performed by 5 clients in 
parallel, 5 clients on any machines should issue a same query with same ID. 
Some clients can join the group during the calculation but before all data in data
nodes are distributed. 

@section Special command
@example
$ paralite test.db "special command"
@end example

special command:
@example
@t{.import FILE|DIR table [record_tag] [-column_separator col_sep] 
                 [-row_separator row_sep]} Import data from FILE into TABLE 
@t{.output FILE}        Send output to FILENAME
@t{.output stdout}	     Send output to the screen
@t{.indices [TABLE]}     Show names of all indices. 
                     If TABLE specified, only show indices for tables
@t{.row_separator STRING}   Change row separator used by output mode and .import
@t{.col_separator STRING}   Change col separator used by output mode and .import
@t{.show}	             Show the current values for various settings
@t{.analyze SQL}       Show the logical plan of a SQL query;  
@end example

@section Configurations for performance
ParaLite allows you to make your own configuration file to control some parameters
of SQLite or query execution plan. If you want to create the file, firstly create
a file called @t{paralite.conf} in the current directory. 
Of course, you have to assure that the super process I mentioned above can access it.
@example
$ cat paralite.conf
[db]
cache_size=-1        ; unit = KB,  -1: use the default size of sqlite
temp_store=0         ; where to store the temporary tables and indices: 
                       0--default, 1--file, 2--memory
[runtime]
worker_num=-1, -1, -1  ; they are the number of worker for operators. 
                         -1: use all data nodes
@end example
The value of worker_num depends on the shape of execution plan for a query.
Before you set it, you probably need to use the analyze command to get the execution
plan:
@example
$ paralite test.db ".analyze 'select F(a) from x, y where x.a = y.b' with F=\"cmd\""
- udx: cmd
  -- join: x.a = y.b
     --- sql1: select x.a from x
     --- sql2: select y.b from y
@end example
So the @t{worker_num} could be -1, -1, 3, -1. You can only give non -1 value to 
operators who are not @i{sql} and @i{udx} since @i{sql} should be executed by 
every data node and @i{udx} is performed by each computing client.

@section Logs, temporary data and databases 
Once a query is issued, you have two directories named @t{.paralite-log} where stores all
logging information and @t{.paralite-tmp} where stores all intermediate data in home
directory by default. You can also specify your own directories for them by specifing 
the options @t{-log} and @t{-temp} respectively.

When you firstly use a database, for instance, when you issue the next SQL:
@example
$ paralite test.db "create table x (a int, b) on hongo[[100-102]]"
@end example
A @t{test.db} is firstly created to store all metadata info and 3 other files named
@t{test.db-hostname-time-id} are created on hongo100, hongo101 and hongo102 respectively in the same directory with @t{test.db}.


@chapter Using ParaLite in Real-World Workflows
@menu
* 
@end menu
TODO

@bye
                                   

